\subsection{Motor model}
ในแต่ละใบพัดของ quadrotor จะให้แรงที่มีทิศทางตรงข้ามกับแกน $Z$ ของ body frame
แรงที่เกิดจากการหมุนจะทำให้เกิดแรงบิดรอบแนวแกน roll pitch yaw โดยแรงและแรงบิดจะขึ้นอยู่กับความเร็วของใบพัดและค่าคงที่ค่าหนึ่ง
ซึ่งสามารถแสดงความสัมพันธ์ของแรงได้ดังสมการที่ \ref{equ:force_equ} และความสัมพันธ์ของแรงบิดได้ดังสมการที่ \ref{equ:torque_equ}

\begin{equation}
    {F_i = k_f\Omega_i^2}
	\label{equ:force_equ}
\end{equation}

\begin{equation}
    {T_i = -k_tsgn(i)\Omega_i^2}
	\label{equ:torque_equ}
\end{equation}

โดยที่ตัวห้อย $i$ จะแสดงถึงลำดับของใบพัดแต่ละตัว $\Omega_i$ แสดงถึงความเร็วการหมุนของใบพัด $i$
$k_f$ และ $k_t$ เป็นค่าคงที่การคูณของแรงและแรงบิด $sgn$ เป็นฟังก์ชันที่ใช้บอกทิศทางการหมุนของใบพัดที่มีความแตกต่างกันตามการติดตั้ง
แรงทั้งหมดที่เกิดบน quadrotor สามารถที่จะเขียนสมการให้อยู่ในรูปของความเร็วการหมุนของใบพัดคูณกับแมทริกการแปลง $M$ ดังสมการที่ \ref{equ:map_force_torque}

\begin{equation}
    {\begin{bmatrix} 
    F \\[10pt] \tau_\phi \\[10pt] \tau_\theta \\[10pt] \tau_\psi \\
    \end{bmatrix} = M\begin{bmatrix}
    \Omega_1^2 \\[10pt]
    \Omega_2^2 \\[10pt]
    \Omega_3^2 \\[10pt]
    \Omega_4^2 \\       
    \end{bmatrix}}
	\label{equ:map_force_torque}
\end{equation}

โดยที่ แมทริกการแปลง $M$ มีการกำหนดไว้ดังสมการที่ \ref{equ:map_matrix}

\begin{equation}
    {M = \begin{bmatrix}
    k_f & k_f & k_f & k_f \\[10pt]
    \frac{k_fd}{\sqrt{2}} & -\frac{k_fd}{\sqrt{2}} & -\frac{k_fd}{\sqrt{2}} & \frac{k_fd}{\sqrt{2}}  \\[10pt]
    \frac{k_fd}{\sqrt{2}} & \frac{k_fd}{\sqrt{2}} & -\frac{k_fd}{\sqrt{2}} & -\frac{k_fd}{\sqrt{2}}  \\[10pt]
    k_t & -k_t & k_t & -k_t \\
    \end{bmatrix}}
	\label{equ:map_matrix}
\end{equation}

โดยที่ $\Omega_i^2$ เป็นความเร็วการหมุนของใบพัดกำลังสอง และ $d$ คือระยะทางจากจุดศูนย์กลางมวลของ quadrotor ไปถึงจุดกึ่งกลางของใบพัด


